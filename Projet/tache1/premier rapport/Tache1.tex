\documentclass[11pt,a4paper]{report}

\usepackage[utf8]{inputenc}
\usepackage[francais]{babel}
\usepackage[T1]{fontenc}
\usepackage{amsmath}
\usepackage{amsfonts}
\usepackage{amssymb}
\usepackage{graphicx}
\usepackage[squaren,Gray]{SIunits}
\usepackage{numprint}
\usepackage{mhchem}


\newcommand{\dsp}{\displaystyle}
\renewcommand\arraystretch{2.5}

\author{Groupe 1246}
\title{Projet P3 LFSAB1503: Rapport de la première tâche}
\begin{document}
\maketitle

\section*{Équation de la réaction et bilan de matière}

Il nous est demandé de rechercher la quantité des différents composés nécessaire à la synthèse de l'ammoniac.
Il nous était dit que l'ammoniac pouvait être obtenu à partir de dihydrogène ($H_2$) et de diazote ($N_2$).
Nous sommes donc arrivés à l'équation de synthèse de l'ammoniac suivante: 


 $$\ce{N_{2(g)} + 3H_{2(g)} -> 2NH_{3(g)}}$$

La masse molaire de l'ammoniac étant de \unit{17}{\gram\per\mole}, nous en avons déduit que une masse de \unit{1000}{\ton}
correspondait à \unit{\dfrac{10^{9}}{17}}{\mole}. Nous avons ensuite fait un tableau d'avancement de la réaction,
où les données sont exprimées en moles.

\begin{figure}[h]
\centering
\begin{tabular}{|c|c|c|c|}
\hline 
 & $N_{2(g)}$ & $3H_{2(g)} $ & $2NH_{3(g)}$ \\ 
\hline 
Initial & $\dfrac{10^{9}}{17} \cdot \dfrac{1}{2}$ & $\dfrac{10^{9}}{17} \cdot \dfrac{3}{2}$ & 0 \\ 
\hline 
Réaction & -$\dfrac{10^{9}}{17} \cdot \dfrac{1}{2}$ & -$\dfrac{10^{9}}{17} \cdot \dfrac{3}{2}$ & +$\dfrac{10^{9}}{17}$ \\ 
\hline 
Final & 0 & 0 & $\dfrac{10^{9}}{17}$ \\ 
\hline 
\end{tabular} 
\caption{Tableau d'avancement de la réaction}
\label{tableau}
\end{figure}

La réaction se produisant en continu, on peut calculer des flux de quantité par seconde.
On obtient selon nos calculs:

\begin{itemize}
  \item{une consommation de $N_2$ égale à: $\dfrac{10^{9}}{17} \cdot \dfrac{1}{2} \cdot \dfrac{1}{3600 \cdot 24} = 340.41 $ \unit{}{\mole\per\second}.}
  \item{une consommation de $H_2$ égale à: $\dfrac{10^{9}}{17} \cdot \dfrac{3}{2} \cdot \dfrac{1}{3600 \cdot 24} = 1021.241 $\unit{}{\mole\per\second}
  \item{une production de $NH_3$ égale à: $\dfrac{10^{9}}{17}} \cdot \dfrac{1}{3600 \cdot 24} = 680.827$ \unit{}{\mole\per\second}
}}\end{itemize}

\section*{Aspect thermique}
Selon nos recherches, nous avons trouvé que la réaction était exothermique ($\Delta H_{react}(\unit{25}{\celsius}) = \unit{-92.2}
{\kilo\joule \per \mole}$ pour la réaction en haut de page). Il nous était indiqué que la température du réacteur devait être maintenue à \unit{500}{\celsius} et que celui-ci, 
vu le caractère exothermique de la réaction, pouvait être refroidi par un débit continu d'eau, dont la température 
variait entre \unit{25}{\celsius} et \unit{90}{\celsius}.

\subsection*{Calcul de volume d'eau nécessaire (pour une mole produite)}

Il nous faut déterminer l'enthalpie de la réaction à \unit{500}{\celsius}, c'est-à-dire \unit{773.15}{\kelvin}.
Nous l'obtenons comme suit:

$$\Delta H(\unit{773.15}{\kelvin}) = \Delta H_{NH_3}(\unit{298.15}{\kelvin}) + \int_{298.15}^{773.15} C_{p, NH_3} dT 
 - \frac{1}{2} \lbrack \Delta H_{N_2}(\unit{298.15}{\kelvin}) $$ 
$$+ \int_{298.15}^{773.15} C_{p, N_2} dT \rbrack - \frac{3}{2}  \lbrack \Delta H_{H_2}(\unit{298.15}{\kelvin}) + \int_{298.15}^{773.15} C_{p, H_2} dT  \rbrack $$

Il est important de préciser que les $C_{p}$ sont les constantes calorifiques massiques des différents composants. 
Nous trouvons leur valeur ainsi que celles des enthalpies dans le livre de référence\footnote{Principes de chimie - P. ATKINS et L.JONES, 2e édition, 2013}.
Nous obtenons finalement une différence d'enthalpie d'approximativement \unit{-57}{\kilo\joule\per\mole} de NH_3.

Nous savons que:
$$q =  C_{p} \cdot \Delta T$$

Au vu des indications données, en supposant que nous travaillons à pression constante, en supposant que la température initiale de
réacteur est de \unit{500}{\celsius}, il vient:
$-57000 \cdot 680.827 = 4.180 \cdot 65 \cdot d_{H_{2}O} \Rightarrow  d_{H_{2}O} = \unit{142830}{\gram\per\second} = \unit{142.830}{\kilogram\per\second}$
Etant donné qu'un kilogramme d'eau représente \unit{1}{\liter}, cela équivaut à  \unit{142.830}{\liter\per\second}.

\section*{Source des réactifs}
\subsection*{Le diazote}
Dans des conditions normales, le diazote est le composant majoritaire de l'air, étant donné qu'il y est présent à 
\numprint{72} \% . Un moyen pour obtenir du diazote est de compresser et refroidir l'air pour arriver à le liquéfier.
Les différents compostants sont ensuite distillés afin de le séparer. Ce procédé est connu sous le nom de "cryogénique".
\footnote{Source: « Société Chimique de France - Le réseau des chimistes ». Consulté le 23 septembre 2014. http://www.societechimiquedefrance.fr/.}
D'autres méthodes sont celle de la perméation gazeuse, ou celle de \textsc{Ramsay}; mais ces méthodes sont nettement moins utilisées
et le diazote résultant est de qualité moindre par rapport au procédé cryogénique.
\subsection*{Le dihydrogène}
Actuellement, la plus grande source de dihydrogène est le reformage de gaz naturel. Le méthane accompagné d'un 
catalyseur vont mener à l'obtention de différents gaz, dont le dihydrogène. C'est malheureusement une technique
qui rejette une quantité de $CO_2$ non-négligeable.

L'électrolyse de l'eau est également une alternative pour la production de dihydrogène. \footnote{« Hydrogène > Air Liquide in BELGIUM and LUXEMBOURG ». Consulté le 23 septembre 2014. http://www.airliquide.be/fr/applications-des-gaz/hydrogene-1.html.}
C'est un moyen respectueux de l'environnement utilisant de l'eau déminéralisée qui sera dissociée au moyen 
d'un courant électrique. Les bulles de gaz formées seront séparées et filtrées, pour arriver à un gaz de bonne qualité.
Mais cette technique ne permet qu'une production en petites quantités, et n'est donc pas tellement utilisée.

\section*{Bilan de matière}
\begin{figure}[ht!]
 \centering
 \includegraphics[scale=0.35]{flowsheet.jpg}
 \caption{Flowsheet production ammoniac}
 \label{scheme}
 
\end{figure}

\end{document}
