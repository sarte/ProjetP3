\section{Annexe}
\subsection{Donn�es et constantes}
On connait les valeurs de $\Delta H\degree (\unit{25}{\celsius})$ pour les �l�ments suivants:

\begin{itemize}
\item{\ce{CH_{4(g)}} $\Rightarrow \unit{-74.6}{\kilo\joule\per\mole}$}
\item{\ce{H_2O_{(g)}} $\Rightarrow \unit{-241.83}{\kilo\joule\per\mole}$}
\item{\ce{CO_{(g)}} $\Rightarrow \unit{-110.53}{\kilo\joule\per\mole}$}
\item{\ce{CO_{2(g)}} $\Rightarrow \unit{-393.51}{\kilo\joule\per\mole}$}
\item{\ce{H_{2(g)}} $\Rightarrow \unit{0}{\kilo\joule\per\mole}$}
\item{\ce{O_{2(g)}} $\Rightarrow \unit{0}{\kilo\joule\per\mole}$}
\end{itemize}

On conna�t les constantes permettant d'utiliser la formule (\ref{eqref:capacite}):

\begin{table}[h]
\centering
\begin{tabular}{|c|c|c|c|c|c|}
\hline 
\rule[-1ex]{0pt}{2.5ex}  & A & B & C & D & E \\ 
\hline 
\rule[-1ex]{0pt}{2.5ex} \ce{CH_{4(g)}} & -0.703 & 108.471 & -42.521 & 5.862 & 0.678 \\ 
\hline 
\rule[-1ex]{0pt}{2.5ex} \ce{H_2O_{(g)}} & 30.092 & 6.832 & 6.793 & -2.534 & 0.082 \\ 
\hline 
\rule[-1ex]{0pt}{2.5ex} \ce{CO_{(g)}} & 25.5675 & 6.0961 & 4.0546 & -2.6713 & 0.131 \\ 
\hline 
\rule[-1ex]{0pt}{2.5ex} \ce{CO_{2(g)}} & 34.2244 & 41.044 & -23.5297 & 5.5352 & -0.129 \\ 
\hline 
\rule[-1ex]{0pt}{2.5ex} \ce{H_{2(g)}} & 33.066 & -11.363 & 11.432 & -2.772 & -0.158 \\ 
\hline 
\end{tabular} 
\caption{Tableau des constantes pour la formule (\ref{eqref:capacite})}
\label{tab:my_label}
\end{table}

\subsection{Code \textsc{MATLAB}}
Voici une fonction en \textsc{MATLAB} cr��e par nos soins, nous permettant de calculer l'enthalpie d'un �l�ment � une certaine
temp�rature $T$, en fonction des diff�rentes constantes utilis�es dans la formule (\ref{eqref:capacite}).
\lstset{language=Matlab,breaklines=true}
\begin{lstlisting}[frame=single]
function [Sol]= HmolT(Hfo,T2,A,B,C,D,E)
%% Hfo est le delta H formation a 298K en kJ/mol, T2 est la temperature pour laquelle on desire connaitre le nouveau delta H, A,B,C,D et E sont des donnees qui doivent etre prises sur le site NIST et qui dependent de la molecule pour laquelle on calcule le nouveau H.
%%
Intcp1 = A*298 + B* (298^2)/2000 + C*(298^3)/(3*1000^2) + D*(298^4)/(4*1000^3) + (-1000^2)*E/298;
= A*T2 + B* (T2^2)/2000 + C*(T2^3)/(3*1000^2) + D*(T2^4)/(4*1000^3) + (-1000^2)*E/T2;
Sol = Hfo*1000 + Intcp2 - Intcp1
end
\end{lstlisting}