
\section{Nombre de tuyaux dans le réacteur primaire de reformage}
Dans cette section, nous allons déterminer le nombre de tuyaux de diamètre valant $d=\unit{10}{\centi\meter}$ que doit comporter 
le réacteur de reformage. On considère que la vitesse superficielle typique à l'entrée du réacteur est de 
\unit{2}{\meter\per\second}. Le résultat que nous allons obtenir dépendra de deux paramètres: la température du réacteur,
et la quantité d'ammoniac produite. En supposant que les gaz sont parfaits, nous pouvons exprimer le flux de $\ce{CH_4}$
et d'$\ce{H_2O}$, grâce à la relation $p\cdot \dot{V}=\dot{n} \cdot R\cdot T$ où $\dot{n}$ est exprimé en 
$[\unit{}{\mole\per\second}]$. Les deux composés se déplaçant simultanément dans les tuyaux, le nombre de tuyaux $N$ se 
trouve via la relation suivante:

$$N = \dfrac{\dot{V}_{\ce{CH_4}} + \dot{V}_{\ce{H_2O}}}{\dot{V}}$$

Grâce à notre outil de calcul de bilan de matière (en Annexes), dans le cadre d'une production journalière de 
\unit{1500}{t} d'ammoniac à \unit{1080}{\kelvin}, nous pouvons déterminer le nombre de tuyaux nécessaires:

$$N = (\dot{n}_{\ce{CH_4}}+\dot{n}_{\ce{H_2O}}) \cdot \dfrac{R \cdot T}{p \cdot \dot{V}}} = (3.8956\cdot 10^7+4.10957\cdot 10^7}}) \cdot \dfrac{8.31451 \cdot 1080}{30\cdot 10^5 \cdot1382.4}} = 173$$

Le résultat est donc \numprint{173} tuyaux.

