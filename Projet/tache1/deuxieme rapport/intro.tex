\section{Introduction}

Ce premier document est une pré-étude au sujet de la production d'ammoniac à partir de méthane. 
Il comprend les bilans de matière et d'énergie du procédé, ainsi qu'un flowsheet simplifié. L'analyse paramétrique 
présentée est basée sur un programme \textsc{Matlab} permettant de déterminer les quantités de réactifs nécessaires
à la production d'une certaine quantité d'ammoniac. Notons que ce programme se basait sur beaucoup d'hypothèses 
simplificatrices; il n'était donc pas optimal lors de la rédaction de cette première tâche. La version disponible en annexe
est une forme améliorée.
Les deux paramètres que nous pouvions faire varier étaient:

\begin{enumerate}
\item la température de sortie du réacteur de reformage à la vapeur de méthane.
\item la quantité d'ammoniac \ce{NH_3} produite par jour.
\end{enumerate}


Nous entamerons ce document par le bilan de matière, pour continuer avec le bilan d'énergie. Nous détaillerons ensuite le nombre de tuyaux
dans le réacteur primaire de reformage, et poursuivrons par l'analyse paramétrique. 
