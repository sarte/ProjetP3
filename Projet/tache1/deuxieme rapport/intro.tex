\section{Introduction}

Dans le cadre de ce premier projet de deuxième année de bachelier en ingénieur civil, il nous a été demandé
de réaliser une pré-étude au sujet de la production d'ammoniac à partir de méthane. Le but est de pouvoir fournir
des données réalistes et des recommandations à une société chimique souhaitant implanter un nouveau site industriel
avant d’entamer le design des installations. Après un petit peu plus de cinq semaines, nous sommes en
mesure de présenter le fruit de notre travail relativement bien entamé. 

Nous nous sommes intéressés à la synthèse d’ammoniac à partir du reformage à la vapeur de méthane. Nous avons
créé un outil de gestion permettant de déterminer l’approvisionnement nécessaire en matière première et énergie, pour
obtenir une certaine capacité d'ammoniac à produire. Il nous fallait donc deux bilans (le bilan de matière et le
bilan d'énergie) ainsi qu'un flow-sheet pour avoir une vision globale des réactions survenant au cours du procédé. Les
deux paramètres avec lesquels nous pouvons faire varier la valeur renvoyée par la fonction que nous avons écrite sont:

\begin{enumerate}
\item la température de sortie du réacteur de reformage à la vapeur de méthane.
\item la capacité de notre machine (ou la quantité d'ammoniac \ce{NH_3} produite par jour).
\end{enumerate}


Ce rapport est la synthèse de notre travail axé sur la première tâche de ce Projet P3. Pour rappel: la Tâche 1 comprend
les bilans de matière et d’énergie du procédé, ainsi qu'un flow-sheet et une analyse paramétrique. Nous entamerons donc ce
document par le bilan d'énergie, pour continuer avec le bilan de matière. Nous détaillerons ensuite le nombre de tuyaux
dans le réacteur primaire de reformage, et poursuivrons par l'analyse paramétrique. Nous terminerons par les annexes, à
savoir, le flow-sheet, les données et les constantes employées lors de nos calculs thermodynamiques, et le code \textsc{MATLAB}.		
