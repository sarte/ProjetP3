\documentclass[11pt,a4paper]{report}
\usepackage[utf8]{inputenc}
\usepackage[francais]{babel}
\usepackage[T1]{fontenc}
\usepackage{amsmath}
\usepackage{amsfonts}
\usepackage{amssymb}
\usepackage{graphicx}
\author{Groupe 1246}
\usepackage[squaren,Gray]{SIunits}
\usepackage{numprint}
\usepackage{mhchem}
\begin{document}
\section*{Bilan d'énergie}
Il nous a été demandé de calculer de faire le bilan énergétique. Sur base du flowsheet simplifié que nous avons complété en S2, nous avons déterminé que seulement trois réactions devait être analyser, celles se produisant dans le bloc du four et dans le réacteur du reformage primaire, afin de réalisé le bilan:
\begin{itemize}
\item{$CH_{4(g)} + 2O_{2(g)} \Rightarrow 2H_{2}O_{(g)} + CO_{2(g)}$ (Combustion de $CH_{4(g)}$, réaction complète)}
\item{$CH_{4(g)} + H_{2}O_{(g)} \Rightarrow 3H_{2(g)} + CO_{(g)}$ (Reformage à vapeur de $CH_{4(g)}$, réaction à l'équilibre)}
\item{$CO_{(g)} + H_{2}O_{(g)} \Rightarrow H_{2(g)} + CO_{2(g)}$ (Reformage à vapeur de $CH_{4}$, réaction à l'équilibre)}
\end{itemize}

Nous avons commencé par calculer le $\Delta H$ de chaque équation.
\subsection*{Le bloc 'Four'}
Dans ce bloc, nous avons traité les réactions suivantes:
\begin{itemize}
\item{$CH_{4(g)} + H_{2}O_{(g)} \Rightarrow 3H_{2(g)} + CO_{(g)}$}
\item{$CO_{(g)} + H_{2}O_{(g)} \Rightarrow H_{2(g)} + CO_{2(g)}$}
\end{itemize}

Nous allons d'abord nous interreser à la première des deux.
Nous savons que:

$\Delta H^o_{react}=3\Delta H^o_{H_2} + \Delta H^o_{CO} - (\Delta H^o_{CH_{4(g)}} + \Delta H^o_{H_{2}O_{(g)}})$

La réaction se passant dans un mileu à une température 'T' donnée, nous devons calculer les différences d'entalpie molaire de formation des composés à cette température, sur base des données expérimentale. Pour cela, nous avons utilisé la relation suivante:

$H^o_m(T_2)=\Delta H^o_{form}(T_1)+\int_{T_1}^{T_2} \Delta C_pdT$   (1)

En faisant des recherches, nous avons trouvé l'expression pour $\Delta C_p$ en fonction de 'T' suivante:

$\Delta C_p=A+\dfrac{BT}{1000}+C(\dfrac{T}{1000})^2+D(\dfrac{T}{1000})^3+\frac{E}{(\dfrac{T}{1000})^2}$        (2)

où T est la température du mileu et A,B,C,D,E sont des constantes propres à chaques composants (cfr Annexe).

De (1) et (2), on trouve facilement:

$H^o_m(T_2)=\Delta H^o_{form}(T_1)+\int_{T_1}^{T_2} [A+\dfrac{BT}{1000}+C(\dfrac{T}{1000})^2+D(\dfrac{T}{1000})^3+\frac{E}{(\dfrac{T}{1000})^2}]$dT

ou encore

$H^o_m(T_2)=\Delta H^o_{form}(T_1) + [AT+\dfrac{BT^2}{2\cdot10^3}+C(\dfrac{T^3}{3\cdot10^6})+D(\dfrac{T^4}{4\cdot10^9})+\frac{-10^6E}{T}]^{T_2}_{T_1}$(3) 

En remplaçant la valeur des $\Delta H^o$ dans (1) de chaque élément par leur valeur à la température 'T' obtenue avec (3), on peut facilement obtenir une expression pour $\Delta H^o_{react}$.


\subsection*{Annexe}
On connait les valeurs de $\Delta H^o$ pour les éléments suivants:
\begin{itemize}
\item{$CH_{4(g)} \Rightarrow \unit{-74.6}{\kilo\joule\per\mole}$}
\item{$H_2O_{(g)} \Rightarrow \unit{-241.83}{\kilo\joule\per\mole}$}
\item{$CO_{(g)} \Rightarrow \unit{-110.53}{\kilo\joule\per\mole}$}
\item{$CO_{2(g)} \Rightarrow \unit{-393.51}{\kilo\joule\per\mole}$}
\end{itemize}

On connaît les constantes permettant d'utiliser la formule (2) suivantes:

\begin{tabular}{|c|c|c|c|c|c|}
\hline 
\rule[-1ex]{0pt}{2.5ex}  & A & B & C & D & E \\ 
\hline 
\rule[-1ex]{0pt}{2.5ex} $CH_{4(g)}$ & -0.703 & 108.471 & -42.521 & 5.862 & 0.678 \\ 
\hline 
\rule[-1ex]{0pt}{2.5ex} $H_2O_{(g)}$ & 30.092 & 6.832 & 6.793 & -2.534 & 0.082 \\ 
\hline 
\rule[-1ex]{0pt}{2.5ex} $CO_{(g)}$ & 25.5675 & 6.0961 & 4.0546 & -2.6713 & 0.131 \\ 
\hline 
\rule[-1ex]{0pt}{2.5ex} $CO_{2(g)}$ & 34.2244 & 41.044 & -23.5297 & 5.5352 & -0.129 \\ 
\hline 
\end{tabular} 
\end{document}