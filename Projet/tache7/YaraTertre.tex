\documentclass{article}
\usepackage[utf8]{inputenc}
\usepackage{natbib}
\usepackage[T1]{fontenc}
\usepackage[francais]{babel}
\usepackage{chemist}
\usepackage{array}
\usepackage[version=3]{mhchem}
\usepackage{amsmath}
\usepackage[squaren,Gray]{SIunits}
\usepackage{numprint}
\usepackage{amsfonts}
\usepackage{amssymb}
\usepackage{graphicx}
\usepackage{mathtools}
\usepackage{fullpage}
\usepackage{mhchem}
\usepackage{listings}
\usepackage{hyperref}
\author{Groupe 1246 \\ Tutrice: Audrey Favache \\ \\
Sparenberg Marie-Charlotte - 5408-13-00\\
Pirson Félix - 6838-13-00\\
Muguerza Bengoechea Gauthier - 7865-13-00\\
Momin Charles - 3920-13-00\\
Hoedenaeken Adrien - 4114-13-00\\
Colpin Lionel - 3965-12-00 \\
Arteaga Michaux Sayri - 3149-12-00\\ }

\begin{document}
\title{Rapport de la tâche 7: Yara-Tertre}
\maketitle
\newpage
\section*{L'entreprise Yara}
L'entreprise Yara est une société chimique multinationale, dont le siège social est basé en Norvège. C'est une des entreprise phare dans la production d'engrais azotés sous forme cristalline. Elle est aussi importante dans le domaine de la production d'ammoniac.
\section*{Le site de Tertre}
On peut se demander pourquoi l'entreprise Yara a choisi Tertre comme
 lieu d'implentation pour son site de production d'ammoniac. Cela est en fait dû à l'emplacement stratégique: l'unité est placée à un croisement de voies marchandes de gaz naturel. En effet, les procédés ayant une consommation moyenne de $\unit{12 \cdot 10^5}{\meter^3\per jour}$ ( la consommation annuelle du centre de Tertre représente $\approx 2\%$ de la consommation annuelle belge), un bon moyen d'apport en gaz naturel n'est pas négligeable. Les principaux pays d'importation sont la Russie et la Norvège.
   La production d'ammoniac n'est pas le domaine principal d'activité sur le site de
    Terte, mais est en fait une partie d'une chaine de production d'engrais sous
     forme de granulés. À titre d'information, le site produit
     environ $\unit{10^6}{\tonne\per an}$ d'engrais. Cela correspond
     à une production moyenne journalière de $NH_3$de $\unit{1100}{\tonne\per jours}$.
      Les batîments étant vétustes, celle quantité de production est limité
       comparée à celle d'une unitée de production de dernière génération qui peut atteindre $\unit{3000-3500}{\tonne\per jours}$. 
\subsection*{Flowsheet simplifié des procédés}
\begin{center}
\includegraphics[scale=0.6]{FlYara.png}
\end{center}

\newpage
\section*{Stockage de l'ammoniac}
\subsection*{Engrais}
Les différents engrais sont stockés sous formes de granulés solides, eux-mêmes stockés sous formes de grands sacs appelés "Big-bag" (\unit{600}{\kilogram}). A titre d'information, lors des journées de grande production, 250 camions sortent de l'usine remplis de big-bags. 
\subsection*{Ammoniac}
Tout d'abord voici quelque propriétés interressante concernant l'ammoniac:

\begin{itemize}
\item{$T_{critique} = \unit{132.4}{\celsius}$}

\item{Point d'ébullition = $\unit{-33.43}{\celsius}$}

\item{Point de fusion = $\unit{-77.76}{\celsius}$}
\end{itemize}

Deux procédés sont utilisés dans le stockage de l'ammoniac: la cryogénisation et la pressurisation. Dans le premier, l'ammoniac produit est refroidit en dessous de $\unit{-33.43}{\celsius}$ pour être obtenu sous forme liquide. Dans le second, l'ammoniac est pressurisé jusque à une pression de $\unit{10-15}{bars}$, à une température de $\unit{20}{\celsius}$. À Tertre, c'est principalement le type de stockage par cryogénisation qui est utilisé. L'ammoniac sous forme aqueuse obtenu est ensuite stocké dans des réservoirs appelés "tank" ayant une capacité de 15000t. De plus, la forme de ces "tanks" est sphérique lorsque le type de stockage utilisé est celui par pressurisation.


\section*{Rendement et utilisation de catalyseurs}
Le rendement moyen de l'unité de production d'ammoniac est de $\approx 15-16\%$ en fin de chaîne. L'utilisation de catalyseur est d'une importance capitale. En effet, celle-ci va considérablement faciliter l'obtention d'un milieu réactionnel propice:

\begin{figure} [h]
\begin{center}
\includegraphics[scale=0.5]{schrend}]
\end{center}
\end{figure}

On remarque que la courbe rouge (celle de l'énergie nécessaire à la réaction avec la présence d'un catalyseur) est nettement inférieure à la noire. L'utilisation d'un catalyseur permet donc de diminuer l'énergie d'activation de la réaction, avec l'avantage de ne pas modifier cette dernière. A titre d'exemple, pour réaliser la réaction de synthèse de l'ammoniac sans catalyseurs afin d'atteindre des même proportions de prodcution que celle de l'unité de production de Tertre, il faudrait que le milieu réactionnel soit porté à une température de $\unit{1000}{\Celsius}$ à une pression de $\unit{2000}{bars}$. Malgré tout, le rendement atteint à Tertre ne le serait pas dans de telles conditions.

\subsubsection*{Energie d'activation}
L'énergie d'activation est l'énergie nécessaire pour qu'une réaction chimique puisse s'établir. Une expression de celle-ci peut être établie en partant de la loi d'Arrhenius:

$k=A\cdot e^{\dfrac{-E_a}{RT}}$

avec A est le coefficientpré-exponentiel, R la constante des gazs parfaits, T la température en Kelvin, $E_a$ l'énergie d'activation et k la constante de vitesse.
On obtient, lorsqu'on travail avec deux températures, grâce un artifice de calculs:

$ln(k)=ln(A)-\dfrac{E_a}{RT} \Rightarrow ln(k_2)-ln(k_1)=\left( ln(A) - \dfrac{E_a}{RT_2} \right) - \left( ln(A) - \dfrac{E_a}{RT_1} \right)$

et au final:

$E_a = \dfrac{\left( ln(\dfrac{k_2}{k_1}) \right) \cdot R}{\dfrac{1}{T_1} - \dfrac{1}{T_2}}$

\subsection*{Explication simplifiée du fonctionnement d'un catalyseur}
Il nous a été expliqué que les catalyseur pouvait être représentés comme des matériaux porreux dont les cavités étaient des endroit réactionnels propice à la réaction:

\begin{figure} [h]
\begin{center}
\includegraphics[scale=0.5]{cata}]
\end{center}
\end{figure}

Les catalyseurs utilisés à Tertre sont formés en général à base de nickel et d'arsenic et on une durée de vie qui est estimée à 12 ans.

\section*{Problèmes et dangers potentiels}
Le centre de production d'ammoniac comporte de multiples dangers potentiels qui doivent être gérés. 

\subsection*{Exothermie des réactions}
Une des difficultée majeure à gérer est l'exothermie de certaines réaction. Par exemple, dans ce cas-ci, celle du réacteur primaire. Au centre de production de Tertre, cette réaction se produit au alentours de $\unit{1000}{Celsius}$. Il a fallu développer des alliages spécifique afin d'assurer la résistance des différentes installations. Un autre avantage des alliages est le fait que ceux-ci résistent mieux à la corrosion hydrogénique qui pourrait apparaitre dans certaines conditions, dont la réaction est la suivante:

$Fe_{3(s)}C+2H_{2(g)} \Rightarrow CH_{4(g)} + 3Fe_{(aq)}$

\subsection*{Arrêt de production et manque de certains composants}
Un problème apparait dans la phase d'arrêt de production: le manque de vapeur sur les catalyseurs. Ceux-ci, quand il ne sont plus en présence de vapeur, vont régir et une couche de carbone va se former à leurs surface. Le matériaux devient une sorte de charbon. Le problème est que en se transformant, ils deviennent une sorte de bouchon dans les tuyaux. La chaîne est alors complétement arrêtée afin d'éviter tout autre problèmes qui pourrait surgir suite à l'apparition de bouchons. La seule façon de pouvoir relancer la mécanique est de remplacer les tuyaux déffectueux. Le problème est que ceux-ci doivent-être commander longtemps à l'avance et coûtent cher à l'unité.

\subsection*{Impact environnemental}
L'impact environnemental du procédé est l'une des préocupation principale
de la société Yara. La production d'ammoniac produit une quantité non négligeable de 
polluants tel que du $\ce CO_2$, de l'arsenic, du nickel, des oxydes d'azote
mais aussi de l'huile provenant des fuites des machines tournantes. 
Contrairement à ce que l'on pourrait penser ce sont les oxydes d'azote
qui polluent le plus, ils sont proportionellement plus néfaste pour la couche 
d'ozone que le $\ce CO_2$, c'est pourquoi Yara est particulièrement attentif au rejet 
de ce composé. Une production journalière de $\unit {1100}{\tonne}$ de $\ce {NH_3}$ génère pas loin de 
$\unit {1000}{\tonne}$ de $\ce CO_2$ dont $50 \%$ sera revendu à des sociétés
tel que ABinbev ou Coca-Cola pour gazéifier leur boissons. Il est évident
que le sol du site sera également contaminer par d'eventuelles fuites d'huile 
ou des produits de nature quelquonque et que cela necessitera donc une depollution du site à 
la fin de l'activitée industriel sur celui-ci. L'utilisation des catalyseurs notamment a base d'arsenic ou de nickel necessite egalement un traimtement 
bien particulier afin de préserver l'environnement. Il est évident que cette liste de polluants est non-exhaustive dans le cadre d'une instalation industrielle 
de la taille de celle de Yara à Tertre,mai par le biais de ces quelques exemples ci-dessus nous pouvons nous rendre comptre de l'impact d'un tels procédé
sur l'environnement et de ce que cela implique au niveau du traitement des déchets.
\end{document}