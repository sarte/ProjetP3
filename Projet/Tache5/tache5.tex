
\section{Pression normale de stockage (\unit{20}{\celsius})}

Comme observé sur la Figure \ref{graph1} la pression normale de stockage à \unit{20}{\celsius} est de \unit{8}{barg} (trait rouge).

\begin{figure}[ht!]
\centering
\includegraphics[scale=0.6]{tache5.PNG}
\caption{Pression $p_1$ vs. Température $T$}
\label{graph1}
\end{figure}

\section{Pression normale de stockage en été (\unit{30}{\celsius})}

De nouveau sur base de la Figure \ref{graph1} nous trouvons que la pression normale de stockage est de \unit{10}{barg} (trait orange).

\section{Pression maximale de tarage de la soupape}

Lors de sa conférence, monsieur \textsc{Minion} nous a expliqué que dans le cas d'un incendie, la pression maximale de tarage valait \numprint{121} \% de la pression de design. Dans notre cas, la pression de design du tank est de \unit{15}{b} ce qui correspond à une pression de tarage de \unit{18}{barg}.

\section{Dimensionnement de la soupape}

Dans cette section nous allons dimensionner la soupape pour la pression de tarage décrite ci-dessus, c'est-à-dire \unit{18}{barg}.
Premièrement, nous avons déterminé que la pression durant la décharge, autrement dit la pression de tarage, était de \unit{18}{barg}. De nouveau grâce à la Figure \ref{graph1} nous avons pu trouver la température de sortie du liquide, à savoir \unit{45}{\celsius}.

Nous devons maintenant calculer la surface nécessaire pour l'orifice de la soupape et grâce à cela déterminer quelle soupape nous devrons commander, parmi celles disponibles sur le marché. Nous nous trouvons dans le cas d'une soupape sur un tank ne contenant qu'une seule phase gazeuse. A partir de ce constat nous pouvons utiliser la formule suivante qui nous permet de trouver la surface de la soupape:

\begin{equation}
A=\dfrac{W}{CK_dP_1K_bK_c}\cdot \sqrt{\dfrac{TZ}{M}}
\label{1}
\end{equation}

où $K_d=0.975$, $K_b=K_c=Z=1$\footnote{Données disponibles sur le site iCampus: \url{http://icampus.uclouvain.be/claroline/backends/download.php?url=L1RhY2hlXzVfZGltZW5zaW9ubmVtZW50X3NvdXBhcGUvTEZTQUIxNTAzX0RpbWVuc2lvbm5lbWVudF9QU1YucGRm&cidReset=true&cidReq=LFSAB1503}, "INTRODUCTION TO PRESSURE SAFETY VALVE (PSV) SIZING", consulté le lundi 8 décembre 2014}, $p_1$ vaut \numprint{121} \% de la pression de tarage (en \bbar), soit \unit{1925.175}{\kilo \pascal}, $T$ est la température pour une pression de \unit{18}{Barg} (voir Figure \ref{graph1}), $M$ la masse molaire de \ce{NH_3} soit \unit{17.031}{\gram \per \mole} et enfin $C$ et $W$ deux constantes à calculer. La première étape consiste donc à calculer $W$, le débit massique par heure de matière à évacuer. Par la relation suivante:

$$W=\dfrac{Q}{\Delta H_{vap}}$$

où $Q=C_1FA_{ws}^{0.82}$ avec $C_1=43200$, $F=1$, le facteur environnemental, et $A_ {ws}$ la surface mouillée \footnote{Calcul "élémentaire" sur base de la forme du tank} qui vaut \unit{48\pi}{\meter ^2} . En partant de là, nous trouvons que $Q=\unit{2779974753}{\joule \per \hour}$.

\begin{figure}[ht!]
\centering
\includegraphics[scale=0.6]{tache51.PNG}
\caption{Enthalpie de vaporisation}
\label{graph2}
\end{figure}

Sur base de la Figure \ref{graph2} nous trouvons que $\Delta H_{vap}=\unit{1125}{\kilo \joule \per \kilo \gram}$ ce qui nous donne un débit de sortie de \unit{2471}{\kilo \gram \per \hour}.
Il nous reste donc à calculer la constante $C$:

$$C=\sqrt{k\left( \dfrac{2}{k+1}\right)^{\left( \dfrac{k+1}{k-1}\right)}}$$

où $K=\dfrac{C_p}{C_v}=1.33$, ce qui nous donne $C=0.02655$. En remplaçant le tout dans la formule \ref{1} on obtient une aire de \unit{2.10}{\centi \metre^2} soit \unit{0.32}{in^2}. Il reste maintenant à trouver dans le tableau (Figure \ref{tab}) à quelle soupape cela correspond. Nous en concluons qu'il nous faudra une soupape de type G.

\begin{figure}[ht!]
\centering
\includegraphics[scale=0.4]{tab.PNG}
\caption{Modèle standard de soupape}
\label{tab}
\end{figure}

\section{Tank avec une pression de design de \unit{20}{Barg}}
Si la pression de design de l’équipement était de 20 barg, le fait d’augmenter la pression de tarage de 5 bar et de la porter à 20 barg aura une influence sur le pression $p_1$, la température $T$ et sur la surface de la soupape.
Il suffit d'appliquer la même procédure que ci-dessus en changeant quelques données. En effet, la pression $p_1$ vaudra dans ce cas \unit{2553.39}{\kilo \pascal}, la température $T$ sera de \unit{50}{\celsius} (Voir Figure\ref{graph1}) et $\Delta H_{vap}=\unit{1100}{\kilo \joule \per \kilo \gram}$, cette valeur étant effectivement influencée par la température (Voir Figure \ref{graph2}). Le calcul de $A$ nous donnera une valeur de \unit{16.6}{\centi \metre^2}.

\section{Conséquence de l'isolation thermique du tank}

Nous revenons au cas du premier calcul de $A$, sauf que dans ce cas c'est le facteur environnemental $F$ qui sera différent et qui sera égal à \numprint{0.132}. Ce qui nous donne $A=\unit{0.28}{\centi \metre^2}$. Ce résultat est logique, car l'élévation de température à l'intérieur sera très faible voire inexistante pour un incendie.
